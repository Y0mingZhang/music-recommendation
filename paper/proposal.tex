\documentclass{article}


\PassOptionsToPackage{numbers, compress}{natbib}
\usepackage[preprint]{neurips_2020}
    
\usepackage[utf8]{inputenc} % allow utf-8 input
\usepackage[T1]{fontenc}    % use 8-bit T1 fonts
\usepackage{hyperref}	    % hyperlinks
\usepackage{url}	    % simple URL typesetting
\usepackage{booktabs}	    % professional-quality tables
\usepackage{amsfonts}	    % blackboard math symbols
\usepackage{nicefrac}	    % compact symbols for 1/2, etc.
\usepackage{microtype}	    % microtypography
\usepackage{xcolor}	    % text color
\usepackage{xspace}
\usepackage{amsmath}
\usepackage{amssymb}

\title{ML Proposal}

% The \author macro works with any number of authors. There are two commands
% used to separate the names and addresses of multiple authors: \And and \AND.
%
% Using \And between authors leaves it to LaTeX to determine where to break the
% lines. Using \AND forces a line break at that point. So, if LaTeX puts 3 of 4
% authors names on the first line, and the last on the second line, try using
% \AND instead of \And before the third author name.

\author{
  Zander Meitus \qquad Yiming Zhang \\
  University of Chicago \\
  \texttt{\{zmeitus,yimingz0\}@uchicago.edu}
}

\newcommand{\aoty}{{\bf AOTY}\xspace}
\DeclareMathOperator{\X}{\mathit{X}}
\DeclareMathOperator{\U}{\mathcal{U}}
\DeclareMathOperator{\I}{\mathcal{I}}
\newcommand{\card}[1]{\ensuremath{\lvert {#1} \rvert}}
\newcommand{\easer}{$\text{EASE}^\text{R}$}
\newcommand{\userknn}{UserKNN\xspace}
\newcommand{\norm}[1]{\ensuremath{\lVert #1 \rVert}}
\begin{document}

\maketitle

\section{Proposal}

% For our project, we would like to research recommender system algorithms
%  applied to music recommendation.
% Recommendation systems are pervasive in the modern technological landscape,
%  including professional (LinkedIn), consumer (Amazon), entertainment (Netflix,
%  Spotify), and even romantic settings (Tinder, Bumble).

\paragraph{Overview}
In this project, we plan to study music recommendation using tools we learned
 in the class.
Our paper assume explicit feedback data, where each user-item interaction
 provides an explicit rating.
For a set of users $\U$ and items $\I$, the user-item matrix $\X$ takes the
 form of $\X = \mathbb{R}^{\card{\U} \times \card{\I}}$, with potentially
 missing values when no rating is available.
Then, the recommendation problem boils down to identifying a set of ``good''
 items $\I_u$ for a user $u$.
In the following paragraphs, we describe the dataset on which we explore music
 recommendation, and two algorithms we plan to implement.
Finally, we outline evaluation metrics that measure the quality of
 recommendation systems.

\paragraph*{Data}
For this project, we have created a dataset of album reviews, named \aoty, by
 scraping data from \url{https://www.albumoftheyear.org}.
On this website, users create profiles and rank albums from 0 to 100.
After filtering out albums with $<5$ total reviews, \aoty has a total of 98k
 albums, and 38k users.
The average album in the dataset has 143.1 reviews.
Since a user is unlikely to have rated all albums, and an album is unlikely to
 be rated by users, the user-album matrix is extremely sparse.
For \aoty, the sparsity $s = 1.4 \times 10^{-3}$.
The {\em data sparsity} problem~\citep{suSurveyCollaborativeFiltering2009} in
 the user-item matrix is a major challenge in most real-world recommendation
 problems, and we believe this property makes \aoty a suitable testbed for
 studying recommendation.

\paragraph*{Methods}
Collaborative Filtering (CF), the technique of inferring the preferences of one
 user using known information about other users, is the dominant class of
 algorithm for recommendation systems.
In this project, we plan to implement, extend and evaluate two CF algorithms,
 \easer and \userknn, demonstrated as among the state-of-the-art in a
 comprehensive study by \citet{anelliTopNRecommendationAlgorithms2022}.

\begin{itemize}
	\item {\em \easer}:
	      \easer~\citep{steckEmbarrassinglyShallowAutoencoders2019}
	      is a linear model parameterized by a item-item matrix $B \in
		      \mathbb{R}^{\card{\I} \times \card{\I}}$.
	      The weights $B$ are optimized with respect to the simple objective $\min_B
		       \norm{\X - X B}_F^2 + \lambda \cdot \norm{B}^2$.
	      The first term minimizes the {\em reconstruction cost}, while the second term
	       is the common $L_2$ regularizer, encouraging $B$ to be low-rank.
	      $B$ admits a degenerate solution $B = I$, under which the
	      recommendation
	      system always recommends items that the user already likes.
	      To avoid such a solution, the author added an additional constraint that
	       $\mathrm{diag}(B) = 0$.

	      \item{\em \userknn}: Neighborhood-based recommendation methods
	      essentially
	      treat the preference of a user $u$ as a weighted sum of
	      preferences of other
	      users under some notion of similarity.
	      For a pair of users $u, v$,
	       \userknn~\citep{resnickGroupLensOpenArchitecture1994} measures the similarity
	       between users with the correlation coefficient $r_{uv}$ between items that $u$
	       and $v$ both rated.
	      The predicted score for an item is the weighted sum of ratings of similar
	       users, after adjusting for the ``baseline rating'' that the user gives to an
	       average item.
\end{itemize}

\paragraph*{Setup and Evaluation}
We are going to evaluate the recommendation systems on a test set with heldout
 users from training.
Beyond common retrieval metrics (e.g., P/R/F1@$k$), we plan to consider the
 following evaluation metrics following
 \citet{anelliTopNRecommendationAlgorithms2022}:  \begin{itemize} \item {\em
             IC}: Item Coverage (IC) measures the percentage of distinct items that are ever
 recorded.
\item {\em Gini}: Gini coefficient of the distribution of recommended items
measures the concentration of recommendation.
\end{itemize}

\newpage
\bibliography{ref,yiming}
\bibliographystyle{abbrvnat}
\end{document}