\documentclass{article}


\PassOptionsToPackage{numbers, compress}{natbib}
\usepackage[final]{neurips_2020}
    
\usepackage[utf8]{inputenc} % allow utf-8 input
\usepackage[T1]{fontenc}    % use 8-bit T1 fonts
\usepackage{hyperref}	    % hyperlinks
\usepackage{url}	    % simple URL typesetting
\usepackage{booktabs}	    % professional-quality tables
\usepackage{amsfonts}	    % blackboard math symbols
\usepackage{nicefrac}	    % compact symbols for 1/2, etc.
\usepackage{microtype}	    % microtypography
\usepackage{xcolor}	    % text color
\usepackage{xspace}

\title{ML Proposal}

% The \author macro works with any number of authors. There are two commands
% used to separate the names and addresses of multiple authors: \And and \AND.
%
% Using \And between authors leaves it to LaTeX to determine where to break the
% lines. Using \AND forces a line break at that point. So, if LaTeX puts 3 of 4
% authors names on the first line, and the last on the second line, try using
% \AND instead of \And before the third author name.

\author{
  Zander Meitus \qquad Yiming Zhang \\
  University of Chicago \\
  \texttt{\{zmeitus,yimingz0\}@uchicago.edu}
}

\newcommand{\aoty}{{\bf AOTY}\xspace}

\begin{document}

\maketitle

\section{Proposal}

For our project, we would like to research into recommender system algorithms
 applied to music recommendation.
Recommendation systems are pervasive in the modern technological landscape,
 including professional (LinkedIn), consumer (Amazon), entertainment (Netflix,
 Spotify), and even romantic settings (dating apps).

\paragraph{Overview}
In this project, we hope to concretely frame the music recommendation problem
 using tools we learned in the class.
For example, one potential framing of music recommendation is regression: given
 the list of preferences $x$ of a user on a set of known albums, we try predict
 the score $y$ that the user would give to a new album.
We will conduct exploratory analysis (e.g., user and album visualization) using
 techniques we learned in class such as Singular Value Decomposition (SVD) will
 be useful.
One interesting question that such a visulization can help answer is how well
 can music genres help summarize a user's preference by identifying clusters in
 the visualization.
We also hope to explore different recomendation modeling using techniques in
 class, suplemented by research from our literature review.
Possible additional areas of interst include recommendation "explainability"
 and challenges of dimensionality reduction in sparse
 matrices~\citep{Afchar_2022,Ling_2021}.

\paragraph*{Data}
We plan to build a dataset of album reviews, named \aoty, from the {\tt
		 albumoftheyear} website.~\footnote{\url{https://www.albumoftheyear.org}} On
 this site, users create profiles and rank albums from 0 to 100.
By scraping data from this website, we can create a dataset of $n$ users and
 $p$ albums.
This matrix will likely have many missing values as a given user is not likely
 to have ranked all albums.
The {\em data sparsity} problem~\citep{suSurveyCollaborativeFiltering2009} in
 the user-item matrix is a real challenge faced by many recommendation systems
 in practice, and we believe this property makes \aoty a suitable testbed for
 our research project.

\paragraph*{Literature Review}
Collaborative Filtering (CF) is a technique of inferring about the preference
 about one user using known information about other users.
For example, if we know that Alice and Bob have similar taste in music (say
 they both like alternative rock and dislike country music), then it seems
 plausible that Bob would enjoy a new album that Alice rates highly.
\citet{herlockerAlgorithmicFrameworkPerforming1999} makes this intuition
 concrete by proposing a non-parametric neighborhood-based method that models
 the preference of a user as a weighted combination of similar users.

Collaborative filtering techniques often assume some notion of ``similarity''
 between users.
Latent Dirichlet Allocation (LDA)~\citep{bleiLatentDirichletAllocation2001} is
 a useful method to measure similarity between users.
In LDA, we model users as a mixture of latent topics (e.g. music genres), and
 measuring similarity between users can be made simple by measuring distance
 between the corresponding mixtures.
Modern collaborative filtering algorithms take advantage of deep learning, and
 utilize deep neural nets~\citep{Anelli_2022}.

\bibliography{ref,yiming}
\bibliographystyle{abbrvnat}
\end{document}