\documentclass{article}


\PassOptionsToPackage{numbers, compress}{natbib}
\usepackage[final]{neurips_2020}
    
\usepackage[utf8]{inputenc} % allow utf-8 input
\usepackage[T1]{fontenc}    % use 8-bit T1 fonts
\usepackage{hyperref}       % hyperlinks
\usepackage{url}            % simple URL typesetting
\usepackage{booktabs}       % professional-quality tables
\usepackage{amsfonts}       % blackboard math symbols
\usepackage{nicefrac}       % compact symbols for 1/2, etc.
\usepackage{microtype}      % microtypography
\usepackage{xcolor}         % text color

\title{ML Proposal}

% The \author macro works with any number of authors. There are two commands
% used to separate the names and addresses of multiple authors: \And and \AND.
%
% Using \And between authors leaves it to LaTeX to determine where to break the
% lines. Using \AND forces a line break at that point. So, if LaTeX puts 3 of 4
% authors names on the first line, and the last on the second line, try using
% \AND instead of \And before the third author name.

\author{
  Zander Meitus \qquad Yiming Zhang \\
  University of Chicago \\
  \texttt{\{zmeitus,yimingz0\}@uchicago.edu}
}

\begin{document}

\maketitle

\section{Proposal}


For our project, we would like to research recommender system algorithms applied 
to music recommendation. Recommendation systems are pervasive in the modern 
technological landscape, including professional (LinkedIn), consumer (Amazon), 
entertainment (Netflix, Spotify), and even romantic settings (dating apps). These
recommendation systems are often powered by collaborative filtering algorithms, 
which can be based on approaches as simple as linear models, or as complex as 
deep neural nets~\citep{Anelli_2022}. 
\textcolor{red}{[Add additional citations?]} \\
\textcolor{red}{[Insert info on music recommendation systems specifically]} \\

Our project plans to use data on album reviews from the website albumoftheyear.org.
Users can create profiles and rank albums from 0 to 100. From this website, we
can create a dataset of $n$ users and $p$ albums. This matrix will likely be
sparse as a given user is not likely to have ranked all albums. We will use this 
dataset to conduct exploratory analysis and album recomendation modeling using
techniques in class, suplemented by research from our literature review.\\
\textcolor{red}{[Do we need more info on project description?]} \\

Possible additional areas of interst include recommendation "explainability" and 
challenges of dimensionality reduction in sparse matrices.~\citep{Afchar_2022}~\citep{Ling_2021}

\bibliography{ref}
\bibliographystyle{abbrvnat}
\end{document}